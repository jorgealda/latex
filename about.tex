\documentclass{beamer}
\usetheme[
          titleline=true,% Show a line below the frame title.
          titlepagelogo=logoUZ,% Logo for the first page.
          logowidth=0.7,
          authorwidth=0.3,
          pageofpages = of
          ]{Zaragoza}

\author{Jorge Alda}
\title{Zaragoza, a pretty theme for \LaTeX{} Beamer}
\subtitle{Quick guide}
\institute{Universidad de Zaragoza}
\date{\today}
\logo{\includegraphics[height=10px]{logoUZ}}

\begin{document}
\begin{frame}[t, plain]
\titlepage
\end{frame}

\begin{frame}[t]{What is this?}
\begin{itemize}
\item Beamer is a \LaTeX{} class that allows you to create presentations
\item The project home page is http://latex-beamer.sourceforge.net/
\item Beamer contains several themes, but they are a bit ugly
  \begin{itemize}
  \item But with a lot of useful features, such as navigation bars, outlines,
        etc.
  \end{itemize}
\item Zaragoza is a pretty theme
  \begin{itemize}
  \item With a lot of useless -- but pretty -- features
  \item But without some useful features
  \item Well suited for short talks, for longer talks you should use themes
        with navigation bars
  \end{itemize}
\item Why the name?
  \begin{itemize}
  \item Other themes are named after locations of Universities or conferences
  \item Zaragoza is the location of my university
  \end{itemize}
\end{itemize}
\end{frame}

\begin{frame}[t,fragile]{How to use the theme}
\begin{itemize}
\item Install Beamer
  \begin{itemize}
  \item Some distros have a \verb!latex-beamer! package
  \end{itemize}
\item Read the Beamer documentation
  \begin{itemize}
  \item \verb!/usr/share/doc/latex-beamer/beameruserguide.pdf.gz! if you are
        using Debian
  \item \verb!doc/beameruserguide.pdf! in the source package
  \end{itemize}
\item Install the theme
  \begin{itemize}
  \item \verb!mkdir -p ~/texmf/tex/latex/beamer!\\
  \item \verb!cp *.sty ~/texmf/tex/latex/beamer!
  \end{itemize}
\item Or just copy all the \verb!*.sty! files on your current folder
\item Read the example files
  \begin{itemize}
  \item \verb!about.tex!: this file.
  \end{itemize}
\end{itemize}
\end{frame}

\begin{frame}[t,fragile]{Theme files}
\begin{itemize}
\item Themes are composed by sub-themes for single features
\item Inner themes define how the title page, the bullet lists, margins,
      etc. work
  \begin{itemize}
    \item \verb!beamerinnerthemefancyz.sty!
  \end{itemize}
\item Outer themes define how headers and footers look like
  \begin{itemize}
    \item \verb!beamerouterthemedecolinesz.sty!
  \end{itemize}
\item Color themes define the colors to be used in outer and inner themes
  \begin{itemize}
    \item \verb!beamercolorthemeunizar.sty!: blue footers, headers and
          frame title
  \end{itemize}
\item Global themes just include inner, outer and color themes
  \begin{itemize}
    \item \verb!beamerthemeZaragoza.sty!
  \end{itemize}
\end{itemize}
\end{frame}

\begin{frame}[t,fragile]{Configuring the theme}
\begin{itemize}
\item Beamer themes can be configured with options between \verb![! and
      \verb!]!
  \begin{itemize}
  \item \verb!\usetheme[option1 = value, option2 = value]{Zaragoza}!
  \end{itemize}
\item If you do not specify any option, you get
  \begin{itemize}
  \item No logo on titlepage
  \item Unizar (blue) color theme
  \item Squares for bullet lists
  \end{itemize}
\item Color themes can be changed with \verb!\usecolortheme!
  \begin{itemize}
  \item \verb!\usecolortheme{chameleon}!: green
  \item \verb!\usecolortheme{crane}!: yellow and blue
  \end{itemize}
\item A logo, shown in the upper right corner, can be choosen with the
      \verb!\logo! command
  \begin{itemize}
  \item \verb!\logo{\includegraphics[height=50px]{logo-image}}!
  \end{itemize}
\end{itemize}
\end{frame}

\begin{frame}[t,fragile]{The title page}
\begin{itemize}
\item You can put a logo in the title page, just pass the file name using the
      \verb!titlepagelogo! option
\item Logo width is controlled by the \verb!logowidth! option, wich sets the proprtion of the pagewidth filled by the logo.
\item In order to display correctly autors and date, you have to use the \verb!authorwidth! and set it to 1-\verb!logowidth! (or less).
\item Remember to use a plain and top-aligned frame when using title
      pages:\\
      \vskip1ex
      \verb!\begin{frame}[t,plain]!\\
      \verb!\titlepage!\\
      \verb!\end{frame}!
\end{itemize}
\end{frame}

\begin{frame}[t,fragile]{Other options}
\begin{itemize}
\item The \verb!pageofpages! option defines the string between the current
      page number and the total page count
  \begin{itemize}
  \item The default is ``/''
  \item The example files set \verb!pageofpages! to ``of''
  \end{itemize}
\item The \verb!bullet! option can be used to choose the symbol used in
      bullet lists
  \begin{itemize}
  \item \verb!square!: A filled square
        ({\usebeamercolor[fg]{item}\tiny\raise0.2ex\hbox{$\blacksquare$}}) for
        first and third level items, an empty square
        ({\usebeamercolor[fg]{item}\tiny\raise0.2ex\hbox{$\square$}}) for
        second level items
  \item \verb!circle!: A filled circle ({\usebeamercolor[fg]{item}$\bullet$})
        for first and third level items, an empty circle
        ({\usebeamercolor[fg]{item}$\circ$}) for second level items
  \item The default value is \verb!square!
  \end{itemize}
\item If the \verb!titleline! option is set to \verb!true!, a horizontal line
      is drawn below the title
\end{itemize}
\end{frame}
\end{document}

